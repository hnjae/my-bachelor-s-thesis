% !TEX root = main.tex
% !TEX encoding = UTF-8

\chapter{Related Work}
\section{DNNExplorer}
There are various studies on customized DNN accelerators using FPGAs.
Zhang and Ye study methods to explorer optimal FPGA design in \cite{zhang2020dnnexplorer}. They develop a framework named DNNExplorer for modeling and exploring a design. They claim that existing FPGA-based DNN accelerators typically fall into two design paradigms. The one is a paradigm that adopts a generic reusable architecture to support various DNN networks. A generic reusable architecture means this design sacrifices further performance and efficiency for generic reuse. The other design paradigm is a paradigm that adopts a layer-wise tailor-made architecture to optimize a layer-specific demand. Each layer in this design is combined into pipeline implementation. This design loses scalability as it has a limited number of DNN layers that can be supported. Moreover, deeper DNN means fewer resources for each layer, which might lead to performance degradation.

The key idea of DNNExplorer is to use both paradigms simultaneously. DNNExplorer utilizes the second's pipeline design for the first up to the split-point (SP) layer and uses the first's generic accelerator for the rest. DNNExplorer explores design architecture to find out optimal split-point, and it maps resources for each design architecture.

% As we discussed in the Introduction, our final goal, including the work in this paper, is to create a tool to automatically implements the network from NNC.
% We believe this approach is a worthy reference for our model's future design exploration.
We believe this approach is an excellent reference for further design exploration in our model.

\section{Other implementations of a DNN Framework}
% Added by Yamamoto
% Xilinx社は,HLSツールのVivado HLSと連携して動作するDNN開発環境として,FINN [10]やFINN-R[11]を発表している.
FINN is one of the DNN framework for an FPGA by Xilinx, Inc \cite{umuroglu2017finn,blott2018finn}.
% パラメータのビット精度を指定して学習を行うことができ,FPGAへの実装までをサポートしている.
This framework targets quantized neural network, and the framework provides a training environment to set bit width.
%また,Xilinx社は,PYNQ (PYNQ: Python productivityfor ZYNQ )*1を発表している.
% これは,ハードウェアプラットフォームであるが,開発環境も公開されており,Python言語を利用してFPGAによるDNN推論することが可能である.
Moreover, PYNQ (Python productivity for ZYNQ)\cite{pynq} is a hardware platform to develop DNN inference applications in Python.
% また,Jupyter notebookによりブラウザ上から容易にFPGAによるDNN推論を実行できる.
% Developers can develop DNN inference more easily because they can develop it on a browser with a Jupyter notebook.
Developers can develop DNN inference more quickly because they can develop it on a browser with a Jupyter notebook.
However, our research differs in that we aim to implement a machine learning model generated in a DNN framework that does not target FPGAs.


% また,Nakaharaらは,GUINNESS (GUI based NeuralNetwork Environment SyntheSizer)を開発した[12][13].
Nakahara et al. developed GUINNESS (GUI-based Neural Network Environment SyntheSizer) for DNN inference on an FPGA \cite{nakahara2017fully,yonekawa2017chip}.
% GUINNESSもChainerによる学習フロントエンドを提供しており,学習から高位合成用C++ソースコードの生成までをサポートしている.
GUINNESS also provides train front-end with Chainer and supports from training to C++ code for HLS.
% 生成されるC++のソースコードは,SDSoC向けであり,Linux上で動作する.
The C++ code is inputted to SDSoC \cite{sdsoc}, then HDL codes are generated for target FPGA.
% 本稿では,Linuxではなく,小規模かつ低オーバヘッドなRTOSと推論器が連携して動作する.
The hardware receives an activation from Linux on CPU; however, it co-work with small-scale and low overhead RTOS.
% そのため,リアルタイム性が要求されるシステムや車載システムにおいては,我々のフレームワークの方が適している可能性がある.
Therefore, our framework has advantages for real-time systems and automotive systems.
